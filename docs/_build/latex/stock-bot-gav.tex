%% Generated by Sphinx.
\def\sphinxdocclass{report}
\documentclass[letterpaper,10pt,english,openany,oneside]{sphinxmanual}
\ifdefined\pdfpxdimen
   \let\sphinxpxdimen\pdfpxdimen\else\newdimen\sphinxpxdimen
\fi \sphinxpxdimen=.75bp\relax
\ifdefined\pdfimageresolution
    \pdfimageresolution= \numexpr \dimexpr1in\relax/\sphinxpxdimen\relax
\fi
%% let collapsible pdf bookmarks panel have high depth per default
\PassOptionsToPackage{bookmarksdepth=5}{hyperref}


\PassOptionsToPackage{warn}{textcomp}

\catcode`^^^^00a0\active\protected\def^^^^00a0{\leavevmode\nobreak\ }
\usepackage{cmap}
\usepackage{fontspec}
\defaultfontfeatures[\rmfamily,\sffamily,\ttfamily]{}
\usepackage{amsmath,amssymb,amstext}
\usepackage{polyglossia}
\setmainlanguage{english}



\setmainfont{FreeSerif}[
  Extension      = .otf,
  UprightFont    = *,
  ItalicFont     = *Italic,
  BoldFont       = *Bold,
  BoldItalicFont = *BoldItalic
]
\setsansfont{FreeSans}[
  Extension      = .otf,
  UprightFont    = *,
  ItalicFont     = *Oblique,
  BoldFont       = *Bold,
  BoldItalicFont = *BoldOblique,
]
\setmonofont{FreeMono}[
  Extension      = .otf,
  UprightFont    = *,
  ItalicFont     = *Oblique,
  BoldFont       = *Bold,
  BoldItalicFont = *BoldOblique,
]



\usepackage[Bjarne]{fncychap}
\usepackage{sphinx}

\fvset{fontsize=\small}
\usepackage{geometry}
\usepackage{tikz}\usetikzlibrary{shapes,positioning}\usepackage{amsmath}

% Include hyperref last.
\usepackage{hyperref}
% Fix anchor placement for figures with captions.
\usepackage{hypcap}% it must be loaded after hyperref.
% Set up styles of URL: it should be placed after hyperref.
\urlstyle{same}

\addto\captionsenglish{\renewcommand{\contentsname}{Contents:}}

\usepackage{sphinxmessages}
\setcounter{tocdepth}{1}



\title{stock\sphinxhyphen{}bot\sphinxhyphen{}gav}
\date{Dec 27, 2022}
\release{0.0.1}
\author{barbarich vi, taranenko gs, tsoi as, burkina es}
\newcommand{\sphinxlogo}{\vbox{}}
\renewcommand{\releasename}{Release}
\makeindex
\begin{document}

\pagestyle{empty}
\sphinxmaketitle
\pagestyle{plain}
\sphinxtableofcontents
\pagestyle{normal}
\phantomsection\label{\detokenize{index::doc}}


\sphinxstepscope


\chapter{stock\sphinxhyphen{}bot}
\label{\detokenize{modules:stock-bot}}\label{\detokenize{modules::doc}}
\sphinxstepscope


\section{simulation package}
\label{\detokenize{simulation:simulation-package}}\label{\detokenize{simulation::doc}}

\subsection{Module contents}
\label{\detokenize{simulation:module-simulation}}\label{\detokenize{simulation:module-contents}}\index{module@\spxentry{module}!simulation@\spxentry{simulation}}\index{simulation@\spxentry{simulation}!module@\spxentry{module}}
\sphinxstepscope


\section{src package}
\label{\detokenize{src:src-package}}\label{\detokenize{src::doc}}

\subsection{Subpackages}
\label{\detokenize{src:subpackages}}
\sphinxstepscope


\subsubsection{src.structures package}
\label{\detokenize{src.structures:src-structures-package}}\label{\detokenize{src.structures::doc}}

\paragraph{Submodules}
\label{\detokenize{src.structures:submodules}}

\paragraph{src.structures.st\_portfolio module}
\label{\detokenize{src.structures:module-src.structures.st_portfolio}}\label{\detokenize{src.structures:src-structures-st-portfolio-module}}\index{module@\spxentry{module}!src.structures.st\_portfolio@\spxentry{src.structures.st\_portfolio}}\index{src.structures.st\_portfolio@\spxentry{src.structures.st\_portfolio}!module@\spxentry{module}}\index{Portfolio (class in src.structures.st\_portfolio)@\spxentry{Portfolio}\spxextra{class in src.structures.st\_portfolio}}

\begin{fulllineitems}
\phantomsection\label{\detokenize{src.structures:src.structures.st_portfolio.Portfolio}}
\pysigstartsignatures
\pysigline{\sphinxbfcode{\sphinxupquote{class\DUrole{w}{  }}}\sphinxbfcode{\sphinxupquote{Portfolio}}}
\pysigstopsignatures
\sphinxAtStartPar
Bases: \sphinxcode{\sphinxupquote{object}}
\index{\_\_init\_\_() (Portfolio method)@\spxentry{\_\_init\_\_()}\spxextra{Portfolio method}}

\begin{fulllineitems}
\phantomsection\label{\detokenize{src.structures:src.structures.st_portfolio.Portfolio.__init__}}
\pysigstartsignatures
\pysiglinewithargsret{\sphinxbfcode{\sphinxupquote{\_\_init\_\_}}}{\emph{\DUrole{n}{init\_balance}\DUrole{o}{=}\DUrole{default_value}{100000}}, \emph{\DUrole{n}{tickers}\DUrole{o}{=}\DUrole{default_value}{None}}, \emph{\DUrole{n}{weights}\DUrole{o}{=}\DUrole{default_value}{None}}, \emph{\DUrole{n}{strategy}\DUrole{o}{=}\DUrole{default_value}{None}}, \emph{\DUrole{n}{type\_process}\DUrole{o}{=}\DUrole{default_value}{'sim'}}}{}
\pysigstopsignatures
\sphinxAtStartPar
Инициализация портфеля
\begin{quote}\begin{description}
\sphinxlineitem{Parameters}\begin{itemize}
\item {} 
\sphinxAtStartPar
\sphinxstyleliteralstrong{\sphinxupquote{init\_balance}} (\sphinxstyleliteralemphasis{\sphinxupquote{{[}}}\sphinxstyleliteralemphasis{\sphinxupquote{<class 'int'>}}\sphinxstyleliteralemphasis{\sphinxupquote{, }}\sphinxstyleliteralemphasis{\sphinxupquote{<class 'float'>}}\sphinxstyleliteralemphasis{\sphinxupquote{{]}}}) – начальный баланс портфеля;

\item {} 
\sphinxAtStartPar
\sphinxstyleliteralstrong{\sphinxupquote{tickers}} (\sphinxstyleliteralemphasis{\sphinxupquote{Optional}}\sphinxstyleliteralemphasis{\sphinxupquote{{[}}}\sphinxstyleliteralemphasis{\sphinxupquote{list}}\sphinxstyleliteralemphasis{\sphinxupquote{{[}}}\sphinxstyleliteralemphasis{\sphinxupquote{str}}\sphinxstyleliteralemphasis{\sphinxupquote{{]}}}\sphinxstyleliteralemphasis{\sphinxupquote{{]}}}) – список тикеров акций, которые будут входить в портфель. Если не указан, то портфель будет
состоять из всех акций moex russia index;

\item {} 
\sphinxAtStartPar
\sphinxstyleliteralstrong{\sphinxupquote{weights}} (\sphinxstyleliteralemphasis{\sphinxupquote{Optional}}\sphinxstyleliteralemphasis{\sphinxupquote{{[}}}\sphinxstyleliteralemphasis{\sphinxupquote{list}}\sphinxstyleliteralemphasis{\sphinxupquote{{[}}}\sphinxstyleliteralemphasis{\sphinxupquote{float}}\sphinxstyleliteralemphasis{\sphinxupquote{{]}}}\sphinxstyleliteralemphasis{\sphinxupquote{{]}}}) – веса акций в портфеле. Если не указаны, то все акции будут иметь одинаковый вес;

\item {} 
\sphinxAtStartPar
\sphinxstyleliteralstrong{\sphinxupquote{strategy}} (\sphinxstyleliteralemphasis{\sphinxupquote{Optional}}\sphinxstyleliteralemphasis{\sphinxupquote{{[}}}\sphinxstyleliteralemphasis{\sphinxupquote{callable}}\sphinxstyleliteralemphasis{\sphinxupquote{{]}}}) – стратегия, которая будет использоваться для обновления портфеля.

\item {} 
\sphinxAtStartPar
\sphinxstyleliteralstrong{\sphinxupquote{type\_process}} (\sphinxstyleliteralemphasis{\sphinxupquote{str}}) – тип процесса, в котором работает портфель. Может быть ‘sim’ или ‘real’, соответственно
симуляция или реальный процесс

\end{itemize}

\end{description}\end{quote}

\end{fulllineitems}

\index{calc\_amount() (Portfolio method)@\spxentry{calc\_amount()}\spxextra{Portfolio method}}

\begin{fulllineitems}
\phantomsection\label{\detokenize{src.structures:src.structures.st_portfolio.Portfolio.calc_amount}}
\pysigstartsignatures
\pysiglinewithargsret{\sphinxbfcode{\sphinxupquote{async\DUrole{w}{  }}}\sphinxbfcode{\sphinxupquote{calc\_amount}}}{\emph{\DUrole{n}{strategy\_response}}}{}
\pysigstopsignatures
\sphinxAtStartPar
Вычисляет количество бумаг для покупки / продажи
\begin{quote}\begin{description}
\sphinxlineitem{Parameters}
\sphinxAtStartPar
\sphinxstyleliteralstrong{\sphinxupquote{strategy\_response}} ({\hyperref[\detokenize{src.structures:src.structures.st_strategies.StrategyResponse}]{\sphinxcrossref{\sphinxstyleliteralemphasis{\sphinxupquote{StrategyResponse}}}}}) – ответ стратегии;

\sphinxlineitem{Returns}
\sphinxAtStartPar
количество бумаг для покупки

\sphinxlineitem{Return type}
\sphinxAtStartPar
float

\end{description}\end{quote}

\end{fulllineitems}

\index{calc\_amount\_buy() (Portfolio method)@\spxentry{calc\_amount\_buy()}\spxextra{Portfolio method}}

\begin{fulllineitems}
\phantomsection\label{\detokenize{src.structures:src.structures.st_portfolio.Portfolio.calc_amount_buy}}
\pysigstartsignatures
\pysiglinewithargsret{\sphinxbfcode{\sphinxupquote{async\DUrole{w}{  }}}\sphinxbfcode{\sphinxupquote{calc\_amount\_buy}}}{\emph{\DUrole{n}{strategy\_response}}}{}
\pysigstopsignatures
\sphinxAtStartPar
Расчет количества свободных денег для покупки
\begin{quote}\begin{description}
\sphinxlineitem{Parameters}
\sphinxAtStartPar
\sphinxstyleliteralstrong{\sphinxupquote{strategy\_response}} ({\hyperref[\detokenize{src.structures:src.structures.st_strategies.StrategyResponse}]{\sphinxcrossref{\sphinxstyleliteralemphasis{\sphinxupquote{StrategyResponse}}}}}) – ответ от стратегии на покупку;

\sphinxlineitem{Returns}
\sphinxAtStartPar
количество свободных денег

\sphinxlineitem{Return type}
\sphinxAtStartPar
float

\end{description}\end{quote}

\end{fulllineitems}

\index{calc\_amount\_sell() (Portfolio method)@\spxentry{calc\_amount\_sell()}\spxextra{Portfolio method}}

\begin{fulllineitems}
\phantomsection\label{\detokenize{src.structures:src.structures.st_portfolio.Portfolio.calc_amount_sell}}
\pysigstartsignatures
\pysiglinewithargsret{\sphinxbfcode{\sphinxupquote{async\DUrole{w}{  }}}\sphinxbfcode{\sphinxupquote{calc\_amount\_sell}}}{\emph{\DUrole{n}{strategy\_response}}}{}
\pysigstopsignatures
\sphinxAtStartPar
Расчет количества бумаг для продажи
\begin{quote}\begin{description}
\sphinxlineitem{Parameters}
\sphinxAtStartPar
\sphinxstyleliteralstrong{\sphinxupquote{strategy\_response}} ({\hyperref[\detokenize{src.structures:src.structures.st_strategies.StrategyResponse}]{\sphinxcrossref{\sphinxstyleliteralemphasis{\sphinxupquote{StrategyResponse}}}}}) – ответ от стратегии на продажу;

\sphinxlineitem{Returns}
\sphinxAtStartPar
количество бумаг

\sphinxlineitem{Return type}
\sphinxAtStartPar
float

\end{description}\end{quote}

\end{fulllineitems}

\index{calc\_shares() (Portfolio static method)@\spxentry{calc\_shares()}\spxextra{Portfolio static method}}

\begin{fulllineitems}
\phantomsection\label{\detokenize{src.structures:src.structures.st_portfolio.Portfolio.calc_shares}}
\pysigstartsignatures
\pysiglinewithargsret{\sphinxbfcode{\sphinxupquote{static\DUrole{w}{  }}}\sphinxbfcode{\sphinxupquote{calc\_shares}}}{\emph{\DUrole{n}{tickers}}}{}
\pysigstopsignatures
\sphinxAtStartPar
Расчет весов акций по ковариации
\begin{quote}\begin{description}
\sphinxlineitem{Parameters}
\sphinxAtStartPar
\sphinxstyleliteralstrong{\sphinxupquote{tickers}} (\sphinxstyleliteralemphasis{\sphinxupquote{list}}\sphinxstyleliteralemphasis{\sphinxupquote{{[}}}\sphinxstyleliteralemphasis{\sphinxupquote{str}}\sphinxstyleliteralemphasis{\sphinxupquote{{]}}}) – датафрейм с ценами закрытия акций;

\sphinxlineitem{Returns}
\sphinxAtStartPar
вектор весов акций

\sphinxlineitem{Return type}
\sphinxAtStartPar
\sphinxstyleemphasis{ndarray}

\end{description}\end{quote}

\end{fulllineitems}

\index{call\_strategy() (Portfolio method)@\spxentry{call\_strategy()}\spxextra{Portfolio method}}

\begin{fulllineitems}
\phantomsection\label{\detokenize{src.structures:src.structures.st_portfolio.Portfolio.call_strategy}}
\pysigstartsignatures
\pysiglinewithargsret{\sphinxbfcode{\sphinxupquote{async\DUrole{w}{  }}}\sphinxbfcode{\sphinxupquote{call\_strategy}}}{}{}
\pysigstopsignatures
\sphinxAtStartPar
Вызывает стратегию
\begin{quote}\begin{description}
\sphinxlineitem{Parameters}
\sphinxAtStartPar
\sphinxstyleliteralstrong{\sphinxupquote{dtime\_now}} – текущее время;

\sphinxlineitem{Returns}
\sphinxAtStartPar
None

\sphinxlineitem{Return type}
\sphinxAtStartPar
{[}<class ‘src.structures.st\_strategies.StrategyResponse’>{]}

\end{description}\end{quote}

\end{fulllineitems}

\index{check\_st\_tp() (Portfolio method)@\spxentry{check\_st\_tp()}\spxextra{Portfolio method}}

\begin{fulllineitems}
\phantomsection\label{\detokenize{src.structures:src.structures.st_portfolio.Portfolio.check_st_tp}}
\pysigstartsignatures
\pysiglinewithargsret{\sphinxbfcode{\sphinxupquote{async\DUrole{w}{  }}}\sphinxbfcode{\sphinxupquote{check\_st\_tp}}}{}{}
\pysigstopsignatures
\sphinxAtStartPar
Проверяет стоп\sphinxhyphen{}лосс и тейк профит
\begin{quote}\begin{description}
\sphinxlineitem{Returns}
\sphinxAtStartPar
продает или покупает бумаги

\end{description}\end{quote}

\end{fulllineitems}

\index{free\_balance (Portfolio property)@\spxentry{free\_balance}\spxextra{Portfolio property}}

\begin{fulllineitems}
\phantomsection\label{\detokenize{src.structures:src.structures.st_portfolio.Portfolio.free_balance}}
\pysigstartsignatures
\pysigline{\sphinxbfcode{\sphinxupquote{property\DUrole{w}{  }}}\sphinxbfcode{\sphinxupquote{free\_balance}}\sphinxbfcode{\sphinxupquote{\DUrole{p}{:}\DUrole{w}{  }float}}}
\pysigstopsignatures
\end{fulllineitems}

\index{full\_balance (Portfolio property)@\spxentry{full\_balance}\spxextra{Portfolio property}}

\begin{fulllineitems}
\phantomsection\label{\detokenize{src.structures:src.structures.st_portfolio.Portfolio.full_balance}}
\pysigstartsignatures
\pysigline{\sphinxbfcode{\sphinxupquote{property\DUrole{w}{  }}}\sphinxbfcode{\sphinxupquote{full\_balance}}\sphinxbfcode{\sphinxupquote{\DUrole{p}{:}\DUrole{w}{  }float}}}
\pysigstopsignatures
\end{fulllineitems}

\index{history (Portfolio property)@\spxentry{history}\spxextra{Portfolio property}}

\begin{fulllineitems}
\phantomsection\label{\detokenize{src.structures:src.structures.st_portfolio.Portfolio.history}}
\pysigstartsignatures
\pysigline{\sphinxbfcode{\sphinxupquote{property\DUrole{w}{  }}}\sphinxbfcode{\sphinxupquote{history}}\sphinxbfcode{\sphinxupquote{\DUrole{p}{:}\DUrole{w}{  }str}}}
\pysigstopsignatures
\end{fulllineitems}

\index{log\_history() (Portfolio method)@\spxentry{log\_history()}\spxextra{Portfolio method}}

\begin{fulllineitems}
\phantomsection\label{\detokenize{src.structures:src.structures.st_portfolio.Portfolio.log_history}}
\pysigstartsignatures
\pysiglinewithargsret{\sphinxbfcode{\sphinxupquote{async\DUrole{w}{  }}}\sphinxbfcode{\sphinxupquote{log\_history}}}{\emph{\DUrole{n}{timestamp}}, \emph{\DUrole{n}{current\_structure}}, \emph{\DUrole{n}{received\_structure}}, \emph{\DUrole{n}{new\_structure}}}{}
\pysigstopsignatures
\sphinxAtStartPar
После покупки логирует историю.
\begin{quote}\begin{description}
\sphinxlineitem{Parameters}\begin{itemize}
\item {} 
\sphinxAtStartPar
\sphinxstyleliteralstrong{\sphinxupquote{timestamp}} (\sphinxstyleliteralemphasis{\sphinxupquote{str}}) – время совершения сделки. Формат ‘\%Y\sphinxhyphen{}\%m\sphinxhyphen{}\%d \%H:\%M:\%S’ или ‘\%Y\sphinxhyphen{}\%m\sphinxhyphen{}\%d’

\item {} 
\sphinxAtStartPar
\sphinxstyleliteralstrong{\sphinxupquote{current\_structure}} ({\hyperref[\detokenize{src.structures:src.structures.st_portfolio.Securities}]{\sphinxcrossref{\sphinxstyleliteralemphasis{\sphinxupquote{Securities}}}}}) – текущее состояние портфеля

\item {} 
\sphinxAtStartPar
\sphinxstyleliteralstrong{\sphinxupquote{received\_structure}} ({\hyperref[\detokenize{src.structures:src.structures.st_portfolio.Securities}]{\sphinxcrossref{\sphinxstyleliteralemphasis{\sphinxupquote{Securities}}}}}) – полученное состояние портфеля

\item {} 
\sphinxAtStartPar
\sphinxstyleliteralstrong{\sphinxupquote{new\_structure}} ({\hyperref[\detokenize{src.structures:src.structures.st_portfolio.Securities}]{\sphinxcrossref{\sphinxstyleliteralemphasis{\sphinxupquote{Securities}}}}}) – обновленное состояние портфеля

\end{itemize}

\end{description}\end{quote}

\end{fulllineitems}

\index{securities (Portfolio property)@\spxentry{securities}\spxextra{Portfolio property}}

\begin{fulllineitems}
\phantomsection\label{\detokenize{src.structures:src.structures.st_portfolio.Portfolio.securities}}
\pysigstartsignatures
\pysigline{\sphinxbfcode{\sphinxupquote{property\DUrole{w}{  }}}\sphinxbfcode{\sphinxupquote{securities}}\sphinxbfcode{\sphinxupquote{\DUrole{p}{:}\DUrole{w}{  }dict}}}
\pysigstopsignatures
\end{fulllineitems}

\index{sell\_all() (Portfolio method)@\spxentry{sell\_all()}\spxextra{Portfolio method}}

\begin{fulllineitems}
\phantomsection\label{\detokenize{src.structures:src.structures.st_portfolio.Portfolio.sell_all}}
\pysigstartsignatures
\pysiglinewithargsret{\sphinxbfcode{\sphinxupquote{async\DUrole{w}{  }}}\sphinxbfcode{\sphinxupquote{sell\_all}}}{\emph{\DUrole{n}{dtime\_now}}}{}
\pysigstopsignatures
\sphinxAtStartPar
Продать все имеющиеся бумаги
\begin{quote}\begin{description}
\sphinxlineitem{Parameters}
\sphinxAtStartPar
\sphinxstyleliteralstrong{\sphinxupquote{dtime\_now}} (\sphinxstyleliteralemphasis{\sphinxupquote{str}}) – текущее время;

\sphinxlineitem{Returns}
\sphinxAtStartPar
None

\sphinxlineitem{Return type}
\sphinxAtStartPar
None

\end{description}\end{quote}

\end{fulllineitems}

\index{type\_process (Portfolio property)@\spxentry{type\_process}\spxextra{Portfolio property}}

\begin{fulllineitems}
\phantomsection\label{\detokenize{src.structures:src.structures.st_portfolio.Portfolio.type_process}}
\pysigstartsignatures
\pysigline{\sphinxbfcode{\sphinxupquote{property\DUrole{w}{  }}}\sphinxbfcode{\sphinxupquote{type\_process}}\sphinxbfcode{\sphinxupquote{\DUrole{p}{:}\DUrole{w}{  }str}}}
\pysigstopsignatures
\end{fulllineitems}

\index{update\_securities() (Portfolio method)@\spxentry{update\_securities()}\spxextra{Portfolio method}}

\begin{fulllineitems}
\phantomsection\label{\detokenize{src.structures:src.structures.st_portfolio.Portfolio.update_securities}}
\pysigstartsignatures
\pysiglinewithargsret{\sphinxbfcode{\sphinxupquote{async\DUrole{w}{  }}}\sphinxbfcode{\sphinxupquote{update\_securities}}}{\emph{\DUrole{o}{*}\DUrole{n}{args}}}{}
\pysigstopsignatures
\sphinxAtStartPar
Обновляет бумаги в портфеле.
\begin{quote}\begin{description}
\sphinxlineitem{Parameters}
\sphinxAtStartPar
\sphinxstyleliteralstrong{\sphinxupquote{args}} ({\hyperref[\detokenize{src.structures:src.structures.st_strategies.StrategyResponse}]{\sphinxcrossref{\sphinxstyleliteralemphasis{\sphinxupquote{StrategyResponse}}}}}) – Если позиция короткая, то количество отрицательное.
Если бумаги были проданы или куплены количество соответственно положительное или отрицательное.
После обновления цена будет перевзвешена в соответствии с количеством. Если мы стоим в короткой
позиции, то покупка просто сократит количество бумаг, а на баланс может поступить положительная
разница, при ее наличии. Если в короткой позиции мы продаем бумаги, то их цена перевзвешивается;

\sphinxlineitem{Returns}
\sphinxAtStartPar
обновляет внутренний портфель

\sphinxlineitem{Return type}
\sphinxAtStartPar
None

\end{description}\end{quote}

\end{fulllineitems}


\end{fulllineitems}

\index{PortfolioHistory (class in src.structures.st\_portfolio)@\spxentry{PortfolioHistory}\spxextra{class in src.structures.st\_portfolio}}

\begin{fulllineitems}
\phantomsection\label{\detokenize{src.structures:src.structures.st_portfolio.PortfolioHistory}}
\pysigstartsignatures
\pysigline{\sphinxbfcode{\sphinxupquote{class\DUrole{w}{  }}}\sphinxbfcode{\sphinxupquote{PortfolioHistory}}}
\pysigstopsignatures
\sphinxAtStartPar
Bases: \sphinxcode{\sphinxupquote{object}}
\index{\_\_init\_\_() (PortfolioHistory method)@\spxentry{\_\_init\_\_()}\spxextra{PortfolioHistory method}}

\begin{fulllineitems}
\phantomsection\label{\detokenize{src.structures:src.structures.st_portfolio.PortfolioHistory.__init__}}
\pysigstartsignatures
\pysiglinewithargsret{\sphinxbfcode{\sphinxupquote{\_\_init\_\_}}}{}{}
\pysigstopsignatures
\sphinxAtStartPar
Конструктор класса, отвечающего за историю по портфелю

\end{fulllineitems}

\index{get\_history() (PortfolioHistory method)@\spxentry{get\_history()}\spxextra{PortfolioHistory method}}

\begin{fulllineitems}
\phantomsection\label{\detokenize{src.structures:src.structures.st_portfolio.PortfolioHistory.get_history}}
\pysigstartsignatures
\pysiglinewithargsret{\sphinxbfcode{\sphinxupquote{get\_history}}}{}{}
\pysigstopsignatures
\end{fulllineitems}

\index{log\_history() (PortfolioHistory method)@\spxentry{log\_history()}\spxextra{PortfolioHistory method}}

\begin{fulllineitems}
\phantomsection\label{\detokenize{src.structures:src.structures.st_portfolio.PortfolioHistory.log_history}}
\pysigstartsignatures
\pysiglinewithargsret{\sphinxbfcode{\sphinxupquote{log\_history}}}{\emph{\DUrole{n}{balance}}, \emph{\DUrole{n}{timestamp}}, \emph{\DUrole{n}{current\_structure}}, \emph{\DUrole{n}{received\_structure}}, \emph{\DUrole{n}{new\_structure}}}{}
\pysigstopsignatures
\sphinxAtStartPar
Логирование состояния портфеля
\begin{quote}\begin{description}
\sphinxlineitem{Parameters}\begin{itemize}
\item {} 
\sphinxAtStartPar
\sphinxstyleliteralstrong{\sphinxupquote{balance}} (\sphinxstyleliteralemphasis{\sphinxupquote{{[}}}\sphinxstyleliteralemphasis{\sphinxupquote{<class 'float'>}}\sphinxstyleliteralemphasis{\sphinxupquote{, }}\sphinxstyleliteralemphasis{\sphinxupquote{<class 'int'>}}\sphinxstyleliteralemphasis{\sphinxupquote{{]}}}) – новый баланс портфеля

\item {} 
\sphinxAtStartPar
\sphinxstyleliteralstrong{\sphinxupquote{timestamp}} (\sphinxstyleliteralemphasis{\sphinxupquote{str}}) – дата и    время изменения

\item {} 
\sphinxAtStartPar
\sphinxstyleliteralstrong{\sphinxupquote{current\_structure}} ({\hyperref[\detokenize{src.structures:src.structures.st_portfolio.Securities}]{\sphinxcrossref{\sphinxstyleliteralemphasis{\sphinxupquote{Securities}}}}}) – текущее состояние портфеля

\item {} 
\sphinxAtStartPar
\sphinxstyleliteralstrong{\sphinxupquote{received\_structure}} ({\hyperref[\detokenize{src.structures:src.structures.st_portfolio.Securities}]{\sphinxcrossref{\sphinxstyleliteralemphasis{\sphinxupquote{Securities}}}}}) – полученное состояние портфеля

\item {} 
\sphinxAtStartPar
\sphinxstyleliteralstrong{\sphinxupquote{new\_structure}} ({\hyperref[\detokenize{src.structures:src.structures.st_portfolio.Securities}]{\sphinxcrossref{\sphinxstyleliteralemphasis{\sphinxupquote{Securities}}}}}) – обновленное состояние портфеля

\end{itemize}

\sphinxlineitem{Returns}
\sphinxAtStartPar
только обновляет историю

\end{description}\end{quote}

\end{fulllineitems}


\end{fulllineitems}

\index{Securities (class in src.structures.st\_portfolio)@\spxentry{Securities}\spxextra{class in src.structures.st\_portfolio}}

\begin{fulllineitems}
\phantomsection\label{\detokenize{src.structures:src.structures.st_portfolio.Securities}}
\pysigstartsignatures
\pysigline{\sphinxbfcode{\sphinxupquote{class\DUrole{w}{  }}}\sphinxbfcode{\sphinxupquote{Securities}}}
\pysigstopsignatures
\sphinxAtStartPar
Bases: \sphinxcode{\sphinxupquote{defaultdict}}
\index{get\_json() (Securities method)@\spxentry{get\_json()}\spxextra{Securities method}}

\begin{fulllineitems}
\phantomsection\label{\detokenize{src.structures:src.structures.st_portfolio.Securities.get_json}}
\pysigstartsignatures
\pysiglinewithargsret{\sphinxbfcode{\sphinxupquote{get\_json}}}{}{}
\pysigstopsignatures
\end{fulllineitems}


\end{fulllineitems}



\paragraph{src.structures.st\_purchase module}
\label{\detokenize{src.structures:module-src.structures.st_purchase}}\label{\detokenize{src.structures:src-structures-st-purchase-module}}\index{module@\spxentry{module}!src.structures.st\_purchase@\spxentry{src.structures.st\_purchase}}\index{src.structures.st\_purchase@\spxentry{src.structures.st\_purchase}!module@\spxentry{module}}\index{DataMessage (class in src.structures.st\_purchase)@\spxentry{DataMessage}\spxextra{class in src.structures.st\_purchase}}

\begin{fulllineitems}
\phantomsection\label{\detokenize{src.structures:src.structures.st_purchase.DataMessage}}
\pysigstartsignatures
\pysigline{\sphinxbfcode{\sphinxupquote{class\DUrole{w}{  }}}\sphinxbfcode{\sphinxupquote{DataMessage}}}
\pysigstopsignatures
\sphinxAtStartPar
Bases: \sphinxcode{\sphinxupquote{object}}

\sphinxAtStartPar
Класс, в котором определена структура сообщения с данными.
\index{\_\_init\_\_() (DataMessage method)@\spxentry{\_\_init\_\_()}\spxextra{DataMessage method}}

\begin{fulllineitems}
\phantomsection\label{\detokenize{src.structures:src.structures.st_purchase.DataMessage.__init__}}
\pysigstartsignatures
\pysiglinewithargsret{\sphinxbfcode{\sphinxupquote{\_\_init\_\_}}}{\emph{\DUrole{n}{message\_code}\DUrole{o}{=}\DUrole{default_value}{0}}, \emph{\DUrole{n}{message\_text}\DUrole{o}{=}\DUrole{default_value}{''}}}{}
\pysigstopsignatures
\sphinxAtStartPar
Конструктор класса
\begin{quote}\begin{description}
\sphinxlineitem{Parameters}\begin{itemize}
\item {} 
\sphinxAtStartPar
\sphinxstyleliteralstrong{\sphinxupquote{message\_code}} (\sphinxstyleliteralemphasis{\sphinxupquote{int}}) – флаг. 0 \sphinxhyphen{} все окей данные есть. 1 \sphinxhyphen{} данных нет, так как скорее всего торги не ведутся.
2 \sphinxhyphen{} тикер не найден;

\item {} 
\sphinxAtStartPar
\sphinxstyleliteralstrong{\sphinxupquote{message\_text}} (\sphinxstyleliteralemphasis{\sphinxupquote{str}}) – сообщение

\end{itemize}

\end{description}\end{quote}

\end{fulllineitems}


\end{fulllineitems}

\index{StockPurchaseProcessMoex (class in src.structures.st\_purchase)@\spxentry{StockPurchaseProcessMoex}\spxextra{class in src.structures.st\_purchase}}

\begin{fulllineitems}
\phantomsection\label{\detokenize{src.structures:src.structures.st_purchase.StockPurchaseProcessMoex}}
\pysigstartsignatures
\pysigline{\sphinxbfcode{\sphinxupquote{class\DUrole{w}{  }}}\sphinxbfcode{\sphinxupquote{StockPurchaseProcessMoex}}}
\pysigstopsignatures
\sphinxAtStartPar
Bases: \sphinxcode{\sphinxupquote{object}}
\index{\_\_init\_\_() (StockPurchaseProcessMoex method)@\spxentry{\_\_init\_\_()}\spxextra{StockPurchaseProcessMoex method}}

\begin{fulllineitems}
\phantomsection\label{\detokenize{src.structures:src.structures.st_purchase.StockPurchaseProcessMoex.__init__}}
\pysigstartsignatures
\pysiglinewithargsret{\sphinxbfcode{\sphinxupquote{\_\_init\_\_}}}{\emph{\DUrole{n}{purchase\_requests}}}{}
\pysigstopsignatures\begin{quote}\begin{description}
\sphinxlineitem{Parameters}
\sphinxAtStartPar
\sphinxstyleliteralstrong{\sphinxupquote{purchase\_requests}} (\sphinxstyleliteralemphasis{\sphinxupquote{list}}\sphinxstyleliteralemphasis{\sphinxupquote{{[}}}{\hyperref[\detokenize{src.structures:src.structures.st_purchase.StockPurchaseRequest}]{\sphinxcrossref{\sphinxstyleliteralemphasis{\sphinxupquote{src.structures.st\_purchase.StockPurchaseRequest}}}}}\sphinxstyleliteralemphasis{\sphinxupquote{{]}}}) – запросы от стратегии

\end{description}\end{quote}

\end{fulllineitems}

\index{calc\_purchase\_quantity() (StockPurchaseProcessMoex static method)@\spxentry{calc\_purchase\_quantity()}\spxextra{StockPurchaseProcessMoex static method}}

\begin{fulllineitems}
\phantomsection\label{\detokenize{src.structures:src.structures.st_purchase.StockPurchaseProcessMoex.calc_purchase_quantity}}
\pysigstartsignatures
\pysiglinewithargsret{\sphinxbfcode{\sphinxupquote{async\DUrole{w}{  }static\DUrole{w}{  }}}\sphinxbfcode{\sphinxupquote{calc\_purchase\_quantity}}}{\emph{\DUrole{n}{market\_price}}, \emph{\DUrole{n}{amt\_assets}}}{}
\pysigstopsignatures
\sphinxAtStartPar
Рассчитывает количество акций, которое можно купить на указанную сумму

\begin{sphinxVerbatim}[commandchars=\\\{\}]
\PYG{g+gp}{\PYGZgt{}\PYGZgt{}\PYGZgt{} }\PYG{k+kn}{import} \PYG{n+nn}{asyncio}
\PYG{g+gp}{\PYGZgt{}\PYGZgt{}\PYGZgt{} }\PYG{n}{asyncio}\PYG{o}{.}\PYG{n}{run}\PYG{p}{(}\PYG{n}{StockPurchaseProcessMoex}\PYG{o}{.}\PYG{n}{calc\PYGZus{}purchase\PYGZus{}quantity}\PYG{p}{(}\PYG{n}{market\PYGZus{}price}\PYG{o}{=}\PYG{l+m+mi}{100}\PYG{p}{,} \PYG{n}{amt\PYGZus{}assets}\PYG{o}{=}\PYG{l+m+mi}{1000}\PYG{p}{)}\PYG{p}{)}
\PYG{g+go}{10}
\PYG{g+gp}{\PYGZgt{}\PYGZgt{}\PYGZgt{} }\PYG{n}{asyncio}\PYG{o}{.}\PYG{n}{run}\PYG{p}{(}\PYG{n}{StockPurchaseProcessMoex}\PYG{o}{.}\PYG{n}{calc\PYGZus{}purchase\PYGZus{}quantity}\PYG{p}{(}\PYG{n}{market\PYGZus{}price}\PYG{o}{=}\PYG{l+m+mi}{100}\PYG{p}{,} \PYG{n}{amt\PYGZus{}assets}\PYG{o}{=}\PYG{l+m+mi}{90}\PYG{p}{)}\PYG{p}{)}
\PYG{g+go}{0}
\PYG{g+gp}{\PYGZgt{}\PYGZgt{}\PYGZgt{} }\PYG{n}{asyncio}\PYG{o}{.}\PYG{n}{run}\PYG{p}{(}\PYG{n}{StockPurchaseProcessMoex}\PYG{o}{.}\PYG{n}{calc\PYGZus{}purchase\PYGZus{}quantity}\PYG{p}{(}\PYG{n}{market\PYGZus{}price}\PYG{o}{=}\PYG{l+m+mi}{100}\PYG{p}{,} \PYG{n}{amt\PYGZus{}assets}\PYG{o}{=}\PYG{o}{\PYGZhy{}}\PYG{l+m+mi}{1000}\PYG{p}{)}\PYG{p}{)}
\PYG{g+go}{10}
\end{sphinxVerbatim}
\begin{quote}\begin{description}
\sphinxlineitem{Parameters}\begin{itemize}
\item {} 
\sphinxAtStartPar
\sphinxstyleliteralstrong{\sphinxupquote{market\_price}} (\sphinxstyleliteralemphasis{\sphinxupquote{float}}) – цена акции;

\item {} 
\sphinxAtStartPar
\sphinxstyleliteralstrong{\sphinxupquote{amt\_assets}} (\sphinxstyleliteralemphasis{\sphinxupquote{float}}) – сумма, которую можно потратить на покупку;

\end{itemize}

\sphinxlineitem{Returns}
\sphinxAtStartPar
словарь с информацией о покупке

\sphinxlineitem{Return type}
\sphinxAtStartPar
int

\end{description}\end{quote}

\end{fulllineitems}

\index{generate\_message() (StockPurchaseProcessMoex static method)@\spxentry{generate\_message()}\spxextra{StockPurchaseProcessMoex static method}}

\begin{fulllineitems}
\phantomsection\label{\detokenize{src.structures:src.structures.st_purchase.StockPurchaseProcessMoex.generate_message}}
\pysigstartsignatures
\pysiglinewithargsret{\sphinxbfcode{\sphinxupquote{async\DUrole{w}{  }static\DUrole{w}{  }}}\sphinxbfcode{\sphinxupquote{generate\_message}}}{\emph{\DUrole{n}{moex\_dict}}}{}
\pysigstopsignatures
\sphinxAtStartPar
Генерирует сообщение о покупке
\begin{quote}\begin{description}
\sphinxlineitem{Parameters}
\sphinxAtStartPar
\sphinxstyleliteralstrong{\sphinxupquote{moex\_dict}} (\sphinxstyleliteralemphasis{\sphinxupquote{dict}}) – словарь с информацией о покупке;

\sphinxlineitem{Returns}
\sphinxAtStartPar
код + сообщение о том, как прошла покупка

\sphinxlineitem{Return type}
\sphinxAtStartPar
{\hyperref[\detokenize{src.structures:src.structures.st_purchase.DataMessage}]{\sphinxcrossref{\sphinxstyleemphasis{DataMessage}}}}

\end{description}\end{quote}

\end{fulllineitems}


\end{fulllineitems}

\index{StockPurchaseRequest (class in src.structures.st\_purchase)@\spxentry{StockPurchaseRequest}\spxextra{class in src.structures.st\_purchase}}

\begin{fulllineitems}
\phantomsection\label{\detokenize{src.structures:src.structures.st_purchase.StockPurchaseRequest}}
\pysigstartsignatures
\pysigline{\sphinxbfcode{\sphinxupquote{class\DUrole{w}{  }}}\sphinxbfcode{\sphinxupquote{StockPurchaseRequest}}}
\pysigstopsignatures
\sphinxAtStartPar
Bases: \sphinxcode{\sphinxupquote{object}}
\index{\_\_init\_\_() (StockPurchaseRequest method)@\spxentry{\_\_init\_\_()}\spxextra{StockPurchaseRequest method}}

\begin{fulllineitems}
\phantomsection\label{\detokenize{src.structures:src.structures.st_purchase.StockPurchaseRequest.__init__}}
\pysigstartsignatures
\pysiglinewithargsret{\sphinxbfcode{\sphinxupquote{\_\_init\_\_}}}{\emph{\DUrole{n}{ticker}}, \emph{\DUrole{n}{type\_action}}, \emph{\DUrole{n}{amt\_assets}}, \emph{\DUrole{n}{price}\DUrole{o}{=}\DUrole{default_value}{None}}, \emph{\DUrole{n}{dtime\_now}\DUrole{o}{=}\DUrole{default_value}{None}}}{}
\pysigstopsignatures\begin{quote}\begin{description}
\sphinxlineitem{Parameters}\begin{itemize}
\item {} 
\sphinxAtStartPar
\sphinxstyleliteralstrong{\sphinxupquote{ticker}} (\sphinxstyleliteralemphasis{\sphinxupquote{str}}) – тикер инструмента;

\item {} 
\sphinxAtStartPar
\sphinxstyleliteralstrong{\sphinxupquote{type\_action}} ({\hyperref[\detokenize{src.structures:src.structures.st_strategies.TypeAction}]{\sphinxcrossref{\sphinxstyleliteralemphasis{\sphinxupquote{TypeAction}}}}}) – тип операции. 0 \sphinxhyphen{} ничего, \sphinxhyphen{}1 \sphinxhyphen{} продажа, 1 \sphinxhyphen{} покупка;

\item {} 
\sphinxAtStartPar
\sphinxstyleliteralstrong{\sphinxupquote{amt\_assets}} (\sphinxstyleliteralemphasis{\sphinxupquote{float}}) – сумма, которая доступна к покупке. Должна быть указана в валюте покупки;

\item {} 
\sphinxAtStartPar
\sphinxstyleliteralstrong{\sphinxupquote{price}} (\sphinxstyleliteralemphasis{\sphinxupquote{Optional}}\sphinxstyleliteralemphasis{\sphinxupquote{{[}}}\sphinxstyleliteralemphasis{\sphinxupquote{float}}\sphinxstyleliteralemphasis{\sphinxupquote{{]}}}) – цена, за которую нужно купить. Если None, то покупка по рыночной цене;

\item {} 
\sphinxAtStartPar
\sphinxstyleliteralstrong{\sphinxupquote{dtime\_now}} (\sphinxstyleliteralemphasis{\sphinxupquote{Optional}}\sphinxstyleliteralemphasis{\sphinxupquote{{[}}}\sphinxstyleliteralemphasis{\sphinxupquote{str}}\sphinxstyleliteralemphasis{\sphinxupquote{{]}}}) – дата и время, когда был совершен запрос на покупку. Если None, то текущее время.
Для симуляций необходимо указывать дату и время, когда был совершен запрос на покупку.
Дата\sphinxhyphen{}время должны быть в формате ‘\%Y\sphinxhyphen{}\%m\sphinxhyphen{}\%d \%H:\%M:\%S’ или ‘\%Y\sphinxhyphen{}\%m\sphinxhyphen{}\%d’

\end{itemize}

\end{description}\end{quote}

\end{fulllineitems}

\index{get\_state() (StockPurchaseRequest method)@\spxentry{get\_state()}\spxextra{StockPurchaseRequest method}}

\begin{fulllineitems}
\phantomsection\label{\detokenize{src.structures:src.structures.st_purchase.StockPurchaseRequest.get_state}}
\pysigstartsignatures
\pysiglinewithargsret{\sphinxbfcode{\sphinxupquote{get\_state}}}{}{}
\pysigstopsignatures
\begin{sphinxVerbatim}[commandchars=\\\{\}]
\PYG{g+gp}{\PYGZgt{}\PYGZgt{}\PYGZgt{} }\PYG{n}{StockPurchaseRequest}\PYG{p}{(}\PYG{l+s+s1}{\PYGZsq{}}\PYG{l+s+s1}{sBer}\PYG{l+s+s1}{\PYGZsq{}}\PYG{p}{,} \PYG{n}{TypeAction}\PYG{o}{.}\PYG{n}{BUY}\PYG{p}{,} \PYG{n}{amt\PYGZus{}assets}\PYG{o}{=}\PYG{l+m+mi}{100}\PYG{p}{,} \PYG{n}{dtime\PYGZus{}now}\PYG{o}{=}\PYG{l+s+s1}{\PYGZsq{}}\PYG{l+s+s1}{2022\PYGZhy{}12\PYGZhy{}17}\PYG{l+s+s1}{\PYGZsq{}}\PYG{p}{)}\PYG{o}{.}\PYG{n}{get\PYGZus{}state}\PYG{p}{(}\PYG{p}{)}
\PYG{g+go}{\PYGZob{}\PYGZsq{}ticker\PYGZsq{}: \PYGZsq{}SBER\PYGZsq{}, \PYGZsq{}type\PYGZus{}action\PYGZsq{}: 1, \PYGZsq{}price\PYGZsq{}: None, \PYGZsq{}amt\PYGZus{}assets\PYGZsq{}: 100, \PYGZsq{}dtime\PYGZus{}now\PYGZsq{}: \PYGZsq{}2022\PYGZhy{}12\PYGZhy{}17\PYGZsq{}\PYGZcb{}}
\end{sphinxVerbatim}
\begin{quote}\begin{description}
\sphinxlineitem{Returns}
\sphinxAtStartPar
Возвращает словарь с информацией о покупке

\sphinxlineitem{Return type}
\sphinxAtStartPar
dict

\end{description}\end{quote}

\end{fulllineitems}


\end{fulllineitems}

\index{StockPurchaseResponse (class in src.structures.st\_purchase)@\spxentry{StockPurchaseResponse}\spxextra{class in src.structures.st\_purchase}}

\begin{fulllineitems}
\phantomsection\label{\detokenize{src.structures:src.structures.st_purchase.StockPurchaseResponse}}
\pysigstartsignatures
\pysigline{\sphinxbfcode{\sphinxupquote{class\DUrole{w}{  }}}\sphinxbfcode{\sphinxupquote{StockPurchaseResponse}}}
\pysigstopsignatures
\sphinxAtStartPar
Bases: \sphinxcode{\sphinxupquote{object}}
\index{\_\_init\_\_() (StockPurchaseResponse method)@\spxentry{\_\_init\_\_()}\spxextra{StockPurchaseResponse method}}

\begin{fulllineitems}
\phantomsection\label{\detokenize{src.structures:src.structures.st_purchase.StockPurchaseResponse.__init__}}
\pysigstartsignatures
\pysiglinewithargsret{\sphinxbfcode{\sphinxupquote{\_\_init\_\_}}}{\emph{\DUrole{n}{message}}, \emph{\DUrole{n}{ticker}}, \emph{\DUrole{n}{market\_price}}, \emph{\DUrole{n}{quantity}}, \emph{\DUrole{n}{lot\_quantity}}, \emph{\DUrole{n}{dt\_purchase}\DUrole{o}{=}\DUrole{default_value}{None}}, \emph{\DUrole{n}{exchange\_fee}\DUrole{o}{=}\DUrole{default_value}{0}}}{}
\pysigstopsignatures\begin{quote}\begin{description}
\sphinxlineitem{Parameters}\begin{itemize}
\item {} 
\sphinxAtStartPar
\sphinxstyleliteralstrong{\sphinxupquote{message}} ({\hyperref[\detokenize{src.structures:src.structures.st_purchase.DataMessage}]{\sphinxcrossref{\sphinxstyleliteralemphasis{\sphinxupquote{DataMessage}}}}}) – код + сообщение о том, как прошла покупка;

\item {} 
\sphinxAtStartPar
\sphinxstyleliteralstrong{\sphinxupquote{ticker}} (\sphinxstyleliteralemphasis{\sphinxupquote{str}}) – Тикер бумаги

\item {} 
\sphinxAtStartPar
\sphinxstyleliteralstrong{\sphinxupquote{market\_price}} (\sphinxstyleliteralemphasis{\sphinxupquote{float}}) – Актуальная цена бумаги на рынке

\item {} 
\sphinxAtStartPar
\sphinxstyleliteralstrong{\sphinxupquote{quantity}} (\sphinxstyleliteralemphasis{\sphinxupquote{int}}) – Количество купленных / проданных бумаг

\item {} 
\sphinxAtStartPar
\sphinxstyleliteralstrong{\sphinxupquote{lot\_quantity}} (\sphinxstyleliteralemphasis{\sphinxupquote{int}}) – Количество лотов, которое было куплено / продано

\item {} 
\sphinxAtStartPar
\sphinxstyleliteralstrong{\sphinxupquote{dt\_purchase}} (\sphinxstyleliteralemphasis{\sphinxupquote{Optional}}\sphinxstyleliteralemphasis{\sphinxupquote{{[}}}\sphinxstyleliteralemphasis{\sphinxupquote{str}}\sphinxstyleliteralemphasis{\sphinxupquote{{]}}}) – Дата и время покупки

\item {} 
\sphinxAtStartPar
\sphinxstyleliteralstrong{\sphinxupquote{exchange\_fee}} (\sphinxstyleliteralemphasis{\sphinxupquote{float}}) – Комиссия брокера

\end{itemize}

\end{description}\end{quote}

\end{fulllineitems}


\end{fulllineitems}



\paragraph{src.structures.st\_securities module}
\label{\detokenize{src.structures:module-src.structures.st_securities}}\label{\detokenize{src.structures:src-structures-st-securities-module}}\index{module@\spxentry{module}!src.structures.st\_securities@\spxentry{src.structures.st\_securities}}\index{src.structures.st\_securities@\spxentry{src.structures.st\_securities}!module@\spxentry{module}}\index{InfoSecurityRequest (class in src.structures.st\_securities)@\spxentry{InfoSecurityRequest}\spxextra{class in src.structures.st\_securities}}

\begin{fulllineitems}
\phantomsection\label{\detokenize{src.structures:src.structures.st_securities.InfoSecurityRequest}}
\pysigstartsignatures
\pysigline{\sphinxbfcode{\sphinxupquote{class\DUrole{w}{  }}}\sphinxbfcode{\sphinxupquote{InfoSecurityRequest}}}
\pysigstopsignatures
\sphinxAtStartPar
Bases: \sphinxcode{\sphinxupquote{object}}
\index{\_\_init\_\_() (InfoSecurityRequest method)@\spxentry{\_\_init\_\_()}\spxextra{InfoSecurityRequest method}}

\begin{fulllineitems}
\phantomsection\label{\detokenize{src.structures:src.structures.st_securities.InfoSecurityRequest.__init__}}
\pysigstartsignatures
\pysiglinewithargsret{\sphinxbfcode{\sphinxupquote{\_\_init\_\_}}}{\emph{\DUrole{n}{ticker}}, \emph{\DUrole{n}{start}}, \emph{\DUrole{n}{end}}, \emph{\DUrole{n}{time\_step}}, \emph{\DUrole{n}{columns}}}{}
\pysigstopsignatures
\sphinxAtStartPar
Конструктор класса для запроса данных с биржи.
\begin{quote}\begin{description}
\sphinxlineitem{Parameters}\begin{itemize}
\item {} 
\sphinxAtStartPar
\sphinxstyleliteralstrong{\sphinxupquote{ticker}} (\sphinxstyleliteralemphasis{\sphinxupquote{str}}) – название бумаги;

\item {} 
\sphinxAtStartPar
\sphinxstyleliteralstrong{\sphinxupquote{start}} (\sphinxstyleliteralemphasis{\sphinxupquote{str}}) – дата начала запроса;

\item {} 
\sphinxAtStartPar
\sphinxstyleliteralstrong{\sphinxupquote{end}} (\sphinxstyleliteralemphasis{\sphinxupquote{str}}) – дата конца запроса;

\item {} 
\sphinxAtStartPar
\sphinxstyleliteralstrong{\sphinxupquote{time\_step}} (\sphinxstyleliteralemphasis{\sphinxupquote{str}}) – биржевой тик;

\item {} 
\sphinxAtStartPar
\sphinxstyleliteralstrong{\sphinxupquote{columns}} (\sphinxstyleliteralemphasis{\sphinxupquote{list}}\sphinxstyleliteralemphasis{\sphinxupquote{{[}}}\sphinxstyleliteralemphasis{\sphinxupquote{str}}\sphinxstyleliteralemphasis{\sphinxupquote{{]}}}) – список столбцов

\end{itemize}

\end{description}\end{quote}

\end{fulllineitems}


\end{fulllineitems}

\index{SecurityState (class in src.structures.st\_securities)@\spxentry{SecurityState}\spxextra{class in src.structures.st\_securities}}

\begin{fulllineitems}
\phantomsection\label{\detokenize{src.structures:src.structures.st_securities.SecurityState}}
\pysigstartsignatures
\pysigline{\sphinxbfcode{\sphinxupquote{class\DUrole{w}{  }}}\sphinxbfcode{\sphinxupquote{SecurityState}}}
\pysigstopsignatures
\sphinxAtStartPar
Bases: \sphinxcode{\sphinxupquote{object}}
\index{\_\_init\_\_() (SecurityState method)@\spxentry{\_\_init\_\_()}\spxextra{SecurityState method}}

\begin{fulllineitems}
\phantomsection\label{\detokenize{src.structures:src.structures.st_securities.SecurityState.__init__}}
\pysigstartsignatures
\pysiglinewithargsret{\sphinxbfcode{\sphinxupquote{\_\_init\_\_}}}{\emph{\DUrole{n}{quantity}\DUrole{o}{=}\DUrole{default_value}{0}}, \emph{\DUrole{n}{price}\DUrole{o}{=}\DUrole{default_value}{0}}, \emph{\DUrole{n}{stop\_loss}\DUrole{o}{=}\DUrole{default_value}{None}}, \emph{\DUrole{n}{take\_profit}\DUrole{o}{=}\DUrole{default_value}{None}}}{}
\pysigstopsignatures
\sphinxAtStartPar
Конструктор класса, который будет хранить состояние конкретной бумаги.

\begin{sphinxadmonition}{note}{Note:}
\sphinxAtStartPar
Стоп лосс и тейк профит пока исполняют заявку по бумаге в полном объеме
\end{sphinxadmonition}
\begin{quote}\begin{description}
\sphinxlineitem{Parameters}\begin{itemize}
\item {} 
\sphinxAtStartPar
\sphinxstyleliteralstrong{\sphinxupquote{quantity}} (\sphinxstyleliteralemphasis{\sphinxupquote{int}}) – количество актива;

\item {} 
\sphinxAtStartPar
\sphinxstyleliteralstrong{\sphinxupquote{stop\_loss}} (\sphinxstyleliteralemphasis{\sphinxupquote{Optional}}\sphinxstyleliteralemphasis{\sphinxupquote{{[}}}\sphinxstyleliteralemphasis{\sphinxupquote{float}}\sphinxstyleliteralemphasis{\sphinxupquote{{]}}}) – Цена с целью ограничить свои убытки;

\item {} 
\sphinxAtStartPar
\sphinxstyleliteralstrong{\sphinxupquote{take\_profit}} (\sphinxstyleliteralemphasis{\sphinxupquote{Optional}}\sphinxstyleliteralemphasis{\sphinxupquote{{[}}}\sphinxstyleliteralemphasis{\sphinxupquote{float}}\sphinxstyleliteralemphasis{\sphinxupquote{{]}}}) – Цена, при которой мы получаем таргетированную выгоду;

\item {} 
\sphinxAtStartPar
\sphinxstyleliteralstrong{\sphinxupquote{price}} (\sphinxstyleliteralemphasis{\sphinxupquote{float}}) – цена актива

\end{itemize}

\end{description}\end{quote}

\end{fulllineitems}

\index{security\_value() (SecurityState method)@\spxentry{security\_value()}\spxextra{SecurityState method}}

\begin{fulllineitems}
\phantomsection\label{\detokenize{src.structures:src.structures.st_securities.SecurityState.security_value}}
\pysigstartsignatures
\pysiglinewithargsret{\sphinxbfcode{\sphinxupquote{security\_value}}}{}{}
\pysigstopsignatures
\sphinxAtStartPar
Возвращает стоимость всех бумаг: quantity * price

\begin{sphinxVerbatim}[commandchars=\\\{\}]
\PYG{g+gp}{\PYGZgt{}\PYGZgt{}\PYGZgt{} }\PYG{n}{SecurityState}\PYG{p}{(}\PYG{n}{quantity}\PYG{o}{=}\PYG{l+m+mi}{2}\PYG{p}{,} \PYG{n}{price}\PYG{o}{=}\PYG{l+m+mf}{12.5}\PYG{p}{)}\PYG{o}{.}\PYG{n}{security\PYGZus{}value}\PYG{p}{(}\PYG{p}{)}
\PYG{g+go}{25.0}
\end{sphinxVerbatim}
\begin{quote}\begin{description}
\sphinxlineitem{Returns}
\sphinxAtStartPar
стоимость всех бумаг

\sphinxlineitem{Return type}
\sphinxAtStartPar
float

\end{description}\end{quote}

\end{fulllineitems}

\index{short\_state() (SecurityState method)@\spxentry{short\_state()}\spxextra{SecurityState method}}

\begin{fulllineitems}
\phantomsection\label{\detokenize{src.structures:src.structures.st_securities.SecurityState.short_state}}
\pysigstartsignatures
\pysiglinewithargsret{\sphinxbfcode{\sphinxupquote{short\_state}}}{}{}
\pysigstopsignatures
\begin{sphinxVerbatim}[commandchars=\\\{\}]
\PYG{g+gp}{\PYGZgt{}\PYGZgt{}\PYGZgt{} }\PYG{n}{security\PYGZus{}state} \PYG{o}{=} \PYG{n}{SecurityState}\PYG{p}{(}\PYG{l+m+mi}{1}\PYG{p}{,} \PYG{l+m+mi}{2}\PYG{p}{,} \PYG{l+m+mi}{3}\PYG{p}{,} \PYG{l+m+mi}{4}\PYG{p}{)}
\PYG{g+gp}{\PYGZgt{}\PYGZgt{}\PYGZgt{} }\PYG{n}{security\PYGZus{}state}\PYG{o}{.}\PYG{n}{short\PYGZus{}state}\PYG{p}{(}\PYG{p}{)}
\PYG{g+go}{\PYGZob{}\PYGZsq{}quantity\PYGZsq{}: 1, \PYGZsq{}price\PYGZsq{}: 2, \PYGZsq{}sl\PYGZsq{}: 3, \PYGZsq{}tp\PYGZsq{}: 4\PYGZcb{}}
\end{sphinxVerbatim}
\begin{quote}\begin{description}
\sphinxlineitem{Returns}
\sphinxAtStartPar
возвращает состояние бумаги без ее истории. Не изменяет объект

\sphinxlineitem{Return type}
\sphinxAtStartPar
{\hyperref[\detokenize{src.structures:src.structures.st_securities.SecurityState}]{\sphinxcrossref{\sphinxstyleemphasis{SecurityState}}}}

\end{description}\end{quote}

\end{fulllineitems}

\index{update\_state() (SecurityState method)@\spxentry{update\_state()}\spxextra{SecurityState method}}

\begin{fulllineitems}
\phantomsection\label{\detokenize{src.structures:src.structures.st_securities.SecurityState.update_state}}
\pysigstartsignatures
\pysiglinewithargsret{\sphinxbfcode{\sphinxupquote{update\_state}}}{\emph{\DUrole{n}{new\_quantity}\DUrole{o}{=}\DUrole{default_value}{0}}, \emph{\DUrole{n}{new\_price}\DUrole{o}{=}\DUrole{default_value}{0}}, \emph{\DUrole{n}{sl}\DUrole{o}{=}\DUrole{default_value}{None}}, \emph{\DUrole{n}{tp}\DUrole{o}{=}\DUrole{default_value}{None}}}{}
\pysigstopsignatures
\sphinxAtStartPar
Обновляет состояние бумаги

\begin{sphinxVerbatim}[commandchars=\\\{\}]
\PYG{g+gp}{\PYGZgt{}\PYGZgt{}\PYGZgt{} }\PYG{n}{security\PYGZus{}state} \PYG{o}{=} \PYG{n}{SecurityState}\PYG{p}{(}\PYG{l+m+mi}{1}\PYG{p}{,} \PYG{l+m+mi}{2}\PYG{p}{,} \PYG{l+m+mi}{3}\PYG{p}{,} \PYG{l+m+mi}{4}\PYG{p}{)}
\PYG{g+gp}{\PYGZgt{}\PYGZgt{}\PYGZgt{} }\PYG{n}{security\PYGZus{}state}\PYG{o}{.}\PYG{n}{update\PYGZus{}state}\PYG{p}{(}\PYG{l+m+mi}{5}\PYG{p}{,} \PYG{l+m+mi}{6}\PYG{p}{)}
\PYG{g+gp}{\PYGZgt{}\PYGZgt{}\PYGZgt{} }\PYG{n}{security\PYGZus{}state}
\PYG{g+go}{\PYGZob{}\PYGZsq{}quantity\PYGZsq{}: 5, \PYGZsq{}price\PYGZsq{}: 6, \PYGZsq{}sl\PYGZsq{}: 3, \PYGZsq{}tp\PYGZsq{}: 4\PYGZcb{}}

\PYG{g+gp}{\PYGZgt{}\PYGZgt{}\PYGZgt{} }\PYG{n}{security\PYGZus{}state}\PYG{o}{.}\PYG{n}{history\PYGZus{}security}
\PYG{g+go}{[\PYGZob{}\PYGZsq{}quantity\PYGZsq{}: 1, \PYGZsq{}price\PYGZsq{}: 2, \PYGZsq{}sl\PYGZsq{}: 3, \PYGZsq{}tp\PYGZsq{}: 4\PYGZcb{}, \PYGZob{}\PYGZsq{}quantity\PYGZsq{}: 5, \PYGZsq{}price\PYGZsq{}: 6, \PYGZsq{}sl\PYGZsq{}: 3, \PYGZsq{}tp\PYGZsq{}: 4\PYGZcb{}]}
\end{sphinxVerbatim}
\begin{quote}\begin{description}
\sphinxlineitem{Parameters}\begin{itemize}
\item {} 
\sphinxAtStartPar
\sphinxstyleliteralstrong{\sphinxupquote{new\_quantity}} (\sphinxstyleliteralemphasis{\sphinxupquote{int}}) – количество актива;

\item {} 
\sphinxAtStartPar
\sphinxstyleliteralstrong{\sphinxupquote{new\_price}} (\sphinxstyleliteralemphasis{\sphinxupquote{float}}) – цена актива

\item {} 
\sphinxAtStartPar
\sphinxstyleliteralstrong{\sphinxupquote{sl}} (\sphinxstyleliteralemphasis{\sphinxupquote{Optional}}\sphinxstyleliteralemphasis{\sphinxupquote{{[}}}\sphinxstyleliteralemphasis{\sphinxupquote{float}}\sphinxstyleliteralemphasis{\sphinxupquote{{]}}}) – Цена с целью ограничить свои убытки;

\item {} 
\sphinxAtStartPar
\sphinxstyleliteralstrong{\sphinxupquote{tp}} (\sphinxstyleliteralemphasis{\sphinxupquote{Optional}}\sphinxstyleliteralemphasis{\sphinxupquote{{[}}}\sphinxstyleliteralemphasis{\sphinxupquote{float}}\sphinxstyleliteralemphasis{\sphinxupquote{{]}}}) – Цена, при которой мы получаем таргетированную выгоду;

\end{itemize}

\sphinxlineitem{Return type}
\sphinxAtStartPar
None

\end{description}\end{quote}

\end{fulllineitems}


\end{fulllineitems}

\index{StockSecurityPrice (class in src.structures.st\_securities)@\spxentry{StockSecurityPrice}\spxextra{class in src.structures.st\_securities}}

\begin{fulllineitems}
\phantomsection\label{\detokenize{src.structures:src.structures.st_securities.StockSecurityPrice}}
\pysigstartsignatures
\pysigline{\sphinxbfcode{\sphinxupquote{class\DUrole{w}{  }}}\sphinxbfcode{\sphinxupquote{StockSecurityPrice}}}
\pysigstopsignatures
\sphinxAtStartPar
Bases: \sphinxcode{\sphinxupquote{object}}
\index{\_\_init\_\_() (StockSecurityPrice method)@\spxentry{\_\_init\_\_()}\spxextra{StockSecurityPrice method}}

\begin{fulllineitems}
\phantomsection\label{\detokenize{src.structures:src.structures.st_securities.StockSecurityPrice.__init__}}
\pysigstartsignatures
\pysiglinewithargsret{\sphinxbfcode{\sphinxupquote{\_\_init\_\_}}}{\emph{\DUrole{n}{ticker}}, \emph{\DUrole{n}{market\_price}}}{}
\pysigstopsignatures
\sphinxAtStartPar
Конструктор класс рыночной цены по бумаге
\begin{quote}\begin{description}
\sphinxlineitem{Parameters}\begin{itemize}
\item {} 
\sphinxAtStartPar
\sphinxstyleliteralstrong{\sphinxupquote{ticker}} (\sphinxstyleliteralemphasis{\sphinxupquote{str}}) – тикер бумаги

\item {} 
\sphinxAtStartPar
\sphinxstyleliteralstrong{\sphinxupquote{market\_price}} (\sphinxstyleliteralemphasis{\sphinxupquote{float}}) – рыночная цена

\end{itemize}

\end{description}\end{quote}

\end{fulllineitems}


\end{fulllineitems}



\paragraph{src.structures.st\_strategies module}
\label{\detokenize{src.structures:module-src.structures.st_strategies}}\label{\detokenize{src.structures:src-structures-st-strategies-module}}\index{module@\spxentry{module}!src.structures.st\_strategies@\spxentry{src.structures.st\_strategies}}\index{src.structures.st\_strategies@\spxentry{src.structures.st\_strategies}!module@\spxentry{module}}\index{BaseStrategy (class in src.structures.st\_strategies)@\spxentry{BaseStrategy}\spxextra{class in src.structures.st\_strategies}}

\begin{fulllineitems}
\phantomsection\label{\detokenize{src.structures:src.structures.st_strategies.BaseStrategy}}
\pysigstartsignatures
\pysigline{\sphinxbfcode{\sphinxupquote{class\DUrole{w}{  }}}\sphinxbfcode{\sphinxupquote{BaseStrategy}}}
\pysigstopsignatures
\sphinxAtStartPar
Bases: \sphinxcode{\sphinxupquote{object}}

\sphinxAtStartPar
Базовый класс стратегии. При создании экземпляра класса, в качестве аргументов передаются:
Необходимые параметры для работы конкретной стратегии.
\index{\_\_init\_\_() (BaseStrategy method)@\spxentry{\_\_init\_\_()}\spxextra{BaseStrategy method}}

\begin{fulllineitems}
\phantomsection\label{\detokenize{src.structures:src.structures.st_strategies.BaseStrategy.__init__}}
\pysigstartsignatures
\pysiglinewithargsret{\sphinxbfcode{\sphinxupquote{\_\_init\_\_}}}{\emph{\DUrole{o}{*}\DUrole{n}{args}}, \emph{\DUrole{o}{**}\DUrole{n}{kwargs}}}{}
\pysigstopsignatures\begin{quote}\begin{description}
\sphinxlineitem{Parameters}\begin{itemize}
\item {} 
\sphinxAtStartPar
\sphinxstyleliteralstrong{\sphinxupquote{args}} – 

\item {} 
\sphinxAtStartPar
\sphinxstyleliteralstrong{\sphinxupquote{kwargs}} – 

\end{itemize}

\end{description}\end{quote}

\end{fulllineitems}

\index{get\_decision() (BaseStrategy method)@\spxentry{get\_decision()}\spxextra{BaseStrategy method}}

\begin{fulllineitems}
\phantomsection\label{\detokenize{src.structures:src.structures.st_strategies.BaseStrategy.get_decision}}
\pysigstartsignatures
\pysiglinewithargsret{\sphinxbfcode{\sphinxupquote{get\_decision}}}{\emph{\DUrole{o}{*}\DUrole{n}{args}}, \emph{\DUrole{o}{**}\DUrole{n}{kwargs}}}{}
\pysigstopsignatures
\sphinxAtStartPar
Метод, который возвращает решение стратегии.
\begin{quote}\begin{description}
\sphinxlineitem{Parameters}\begin{itemize}
\item {} 
\sphinxAtStartPar
\sphinxstyleliteralstrong{\sphinxupquote{args}} – 

\item {} 
\sphinxAtStartPar
\sphinxstyleliteralstrong{\sphinxupquote{kwargs}} – 

\end{itemize}

\sphinxlineitem{Returns}
\sphinxAtStartPar
StrategyResponse

\sphinxlineitem{Return type}
\sphinxAtStartPar
{\hyperref[\detokenize{src.structures:src.structures.st_strategies.StrategyResponse}]{\sphinxcrossref{\sphinxstyleemphasis{StrategyResponse}}}}

\end{description}\end{quote}

\end{fulllineitems}

\index{get\_request() (BaseStrategy method)@\spxentry{get\_request()}\spxextra{BaseStrategy method}}

\begin{fulllineitems}
\phantomsection\label{\detokenize{src.structures:src.structures.st_strategies.BaseStrategy.get_request}}
\pysigstartsignatures
\pysiglinewithargsret{\sphinxbfcode{\sphinxupquote{get\_request}}}{\emph{\DUrole{o}{*}\DUrole{n}{args}}, \emph{\DUrole{o}{**}\DUrole{n}{kwargs}}}{}
\pysigstopsignatures
\sphinxAtStartPar
Метод, который возвращает запрос стратегии.
\begin{quote}\begin{description}
\sphinxlineitem{Parameters}\begin{itemize}
\item {} 
\sphinxAtStartPar
\sphinxstyleliteralstrong{\sphinxupquote{args}} – 

\item {} 
\sphinxAtStartPar
\sphinxstyleliteralstrong{\sphinxupquote{kwargs}} – 

\end{itemize}

\sphinxlineitem{Returns}
\sphinxAtStartPar
DataRequest

\sphinxlineitem{Return type}
\sphinxAtStartPar
{\hyperref[\detokenize{src.structures:src.structures.st_strategies.DataRequest}]{\sphinxcrossref{\sphinxstyleemphasis{DataRequest}}}}

\end{description}\end{quote}

\end{fulllineitems}


\end{fulllineitems}

\index{DataRequest (class in src.structures.st\_strategies)@\spxentry{DataRequest}\spxextra{class in src.structures.st\_strategies}}

\begin{fulllineitems}
\phantomsection\label{\detokenize{src.structures:src.structures.st_strategies.DataRequest}}
\pysigstartsignatures
\pysigline{\sphinxbfcode{\sphinxupquote{class\DUrole{w}{  }}}\sphinxbfcode{\sphinxupquote{DataRequest}}}
\pysigstopsignatures
\sphinxAtStartPar
Bases: \sphinxcode{\sphinxupquote{object}}
\index{\_\_init\_\_() (DataRequest method)@\spxentry{\_\_init\_\_()}\spxextra{DataRequest method}}

\begin{fulllineitems}
\phantomsection\label{\detokenize{src.structures:src.structures.st_strategies.DataRequest.__init__}}
\pysigstartsignatures
\pysiglinewithargsret{\sphinxbfcode{\sphinxupquote{\_\_init\_\_}}}{\emph{\DUrole{n}{tickers}\DUrole{o}{=}\DUrole{default_value}{None}}, \emph{\DUrole{n}{dt\_start}\DUrole{o}{=}\DUrole{default_value}{None}}, \emph{\DUrole{n}{dt\_end}\DUrole{o}{=}\DUrole{default_value}{None}}, \emph{\DUrole{n}{dt\_frequency}\DUrole{o}{=}\DUrole{default_value}{None}}, \emph{\DUrole{n}{columns}\DUrole{o}{=}\DUrole{default_value}{('TRADEDATE', 'CLOSE')}}}{}
\pysigstopsignatures
\sphinxAtStartPar
Конструктор базового класса запроса стратегии.
\begin{quote}\begin{description}
\sphinxlineitem{Parameters}\begin{itemize}
\item {} 
\sphinxAtStartPar
\sphinxstyleliteralstrong{\sphinxupquote{tickers}} (\sphinxstyleliteralemphasis{\sphinxupquote{Optional}}\sphinxstyleliteralemphasis{\sphinxupquote{{[}}}\sphinxstyleliteralemphasis{\sphinxupquote{list}}\sphinxstyleliteralemphasis{\sphinxupquote{{[}}}\sphinxstyleliteralemphasis{\sphinxupquote{str}}\sphinxstyleliteralemphasis{\sphinxupquote{{]}}}\sphinxstyleliteralemphasis{\sphinxupquote{{]}}}) – список тикеров, для которых нужно получить данные

\item {} 
\sphinxAtStartPar
\sphinxstyleliteralstrong{\sphinxupquote{dt\_start}} (\sphinxstyleliteralemphasis{\sphinxupquote{Optional}}\sphinxstyleliteralemphasis{\sphinxupquote{{[}}}\sphinxstyleliteralemphasis{\sphinxupquote{Union}}\sphinxstyleliteralemphasis{\sphinxupquote{{[}}}\sphinxstyleliteralemphasis{\sphinxupquote{Timestamp}}\sphinxstyleliteralemphasis{\sphinxupquote{, }}\sphinxstyleliteralemphasis{\sphinxupquote{str}}\sphinxstyleliteralemphasis{\sphinxupquote{{]}}}\sphinxstyleliteralemphasis{\sphinxupquote{{]}}}) – дата и время начала периода

\item {} 
\sphinxAtStartPar
\sphinxstyleliteralstrong{\sphinxupquote{dt\_end}} (\sphinxstyleliteralemphasis{\sphinxupquote{Optional}}\sphinxstyleliteralemphasis{\sphinxupquote{{[}}}\sphinxstyleliteralemphasis{\sphinxupquote{Union}}\sphinxstyleliteralemphasis{\sphinxupquote{{[}}}\sphinxstyleliteralemphasis{\sphinxupquote{Timestamp}}\sphinxstyleliteralemphasis{\sphinxupquote{, }}\sphinxstyleliteralemphasis{\sphinxupquote{str}}\sphinxstyleliteralemphasis{\sphinxupquote{{]}}}\sphinxstyleliteralemphasis{\sphinxupquote{{]}}}) – дата и время конца периода

\item {} 
\sphinxAtStartPar
\sphinxstyleliteralstrong{\sphinxupquote{dt\_frequency}} (\sphinxstyleliteralemphasis{\sphinxupquote{Optional}}\sphinxstyleliteralemphasis{\sphinxupquote{{[}}}\sphinxstyleliteralemphasis{\sphinxupquote{str}}\sphinxstyleliteralemphasis{\sphinxupquote{{]}}}) – частота данных

\item {} 
\sphinxAtStartPar
\sphinxstyleliteralstrong{\sphinxupquote{columns}} (\sphinxstyleliteralemphasis{\sphinxupquote{list}}\sphinxstyleliteralemphasis{\sphinxupquote{{[}}}\sphinxstyleliteralemphasis{\sphinxupquote{str}}\sphinxstyleliteralemphasis{\sphinxupquote{{]}}}) – необходимые колонки, по умолчанию дата\sphinxhyphen{}время + цена закрытия

\end{itemize}

\end{description}\end{quote}

\end{fulllineitems}

\index{get\_json() (DataRequest method)@\spxentry{get\_json()}\spxextra{DataRequest method}}

\begin{fulllineitems}
\phantomsection\label{\detokenize{src.structures:src.structures.st_strategies.DataRequest.get_json}}
\pysigstartsignatures
\pysiglinewithargsret{\sphinxbfcode{\sphinxupquote{get\_json}}}{}{}
\pysigstopsignatures
\sphinxAtStartPar
Метод, который возвращает json\sphinxhyphen{}представление класса
\begin{quote}\begin{description}
\sphinxlineitem{Returns}
\sphinxAtStartPar
json\sphinxhyphen{}представление класса

\end{description}\end{quote}

\end{fulllineitems}


\end{fulllineitems}

\index{StrategyResponse (class in src.structures.st\_strategies)@\spxentry{StrategyResponse}\spxextra{class in src.structures.st\_strategies}}

\begin{fulllineitems}
\phantomsection\label{\detokenize{src.structures:src.structures.st_strategies.StrategyResponse}}
\pysigstartsignatures
\pysigline{\sphinxbfcode{\sphinxupquote{class\DUrole{w}{  }}}\sphinxbfcode{\sphinxupquote{StrategyResponse}}}
\pysigstopsignatures
\sphinxAtStartPar
Bases: \sphinxcode{\sphinxupquote{object}}

\sphinxAtStartPar
Класс, в котором определена общая структура ответа от стратегии
\index{\_\_init\_\_() (StrategyResponse method)@\spxentry{\_\_init\_\_()}\spxextra{StrategyResponse method}}

\begin{fulllineitems}
\phantomsection\label{\detokenize{src.structures:src.structures.st_strategies.StrategyResponse.__init__}}
\pysigstartsignatures
\pysiglinewithargsret{\sphinxbfcode{\sphinxupquote{\_\_init\_\_}}}{\emph{\DUrole{n}{ticker}\DUrole{o}{=}\DUrole{default_value}{None}}, \emph{\DUrole{n}{type\_action}\DUrole{o}{=}\DUrole{default_value}{0}}, \emph{\DUrole{n}{price}\DUrole{o}{=}\DUrole{default_value}{None}}, \emph{\DUrole{n}{quantity}\DUrole{o}{=}\DUrole{default_value}{None}}, \emph{\DUrole{n}{dtime\_now}\DUrole{o}{=}\DUrole{default_value}{None}}, \emph{\DUrole{n}{stop\_loss}\DUrole{o}{=}\DUrole{default_value}{None}}, \emph{\DUrole{n}{take\_profit}\DUrole{o}{=}\DUrole{default_value}{None}}, \emph{\DUrole{n}{comment}\DUrole{o}{=}\DUrole{default_value}{None}}}{}
\pysigstopsignatures
\sphinxAtStartPar
Конструктор класса
\begin{quote}\begin{description}
\sphinxlineitem{Parameters}\begin{itemize}
\item {} 
\sphinxAtStartPar
\sphinxstyleliteralstrong{\sphinxupquote{ticker}} (\sphinxstyleliteralemphasis{\sphinxupquote{Optional}}\sphinxstyleliteralemphasis{\sphinxupquote{{[}}}\sphinxstyleliteralemphasis{\sphinxupquote{str}}\sphinxstyleliteralemphasis{\sphinxupquote{{]}}}) – название инструмента / стратегии

\item {} 
\sphinxAtStartPar
\sphinxstyleliteralstrong{\sphinxupquote{type\_action}} ({\hyperref[\detokenize{src.structures:src.structures.st_strategies.TypeAction}]{\sphinxcrossref{\sphinxstyleliteralemphasis{\sphinxupquote{TypeAction}}}}}) – флаг действия. 1 \sphinxhyphen{} покупка, \sphinxhyphen{}1 \sphinxhyphen{} продажа, 0 \sphinxhyphen{} ничего не делать

\item {} 
\sphinxAtStartPar
\sphinxstyleliteralstrong{\sphinxupquote{price}} (\sphinxstyleliteralemphasis{\sphinxupquote{Optional}}\sphinxstyleliteralemphasis{\sphinxupquote{{[}}}\sphinxstyleliteralemphasis{\sphinxupquote{float}}\sphinxstyleliteralemphasis{\sphinxupquote{{]}}}) – цена

\item {} 
\sphinxAtStartPar
\sphinxstyleliteralstrong{\sphinxupquote{quantity}} (\sphinxstyleliteralemphasis{\sphinxupquote{Optional}}\sphinxstyleliteralemphasis{\sphinxupquote{{[}}}\sphinxstyleliteralemphasis{\sphinxupquote{int}}\sphinxstyleliteralemphasis{\sphinxupquote{{]}}}) – количество акций

\item {} 
\sphinxAtStartPar
\sphinxstyleliteralstrong{\sphinxupquote{dtime\_now}} (\sphinxstyleliteralemphasis{\sphinxupquote{Optional}}\sphinxstyleliteralemphasis{\sphinxupquote{{[}}}\sphinxstyleliteralemphasis{\sphinxupquote{str}}\sphinxstyleliteralemphasis{\sphinxupquote{{]}}}) – дата и время

\item {} 
\sphinxAtStartPar
\sphinxstyleliteralstrong{\sphinxupquote{stop\_loss}} (\sphinxstyleliteralemphasis{\sphinxupquote{Optional}}\sphinxstyleliteralemphasis{\sphinxupquote{{[}}}\sphinxstyleliteralemphasis{\sphinxupquote{float}}\sphinxstyleliteralemphasis{\sphinxupquote{{]}}}) – стоп\sphinxhyphen{}лосс

\item {} 
\sphinxAtStartPar
\sphinxstyleliteralstrong{\sphinxupquote{take\_profit}} (\sphinxstyleliteralemphasis{\sphinxupquote{Optional}}\sphinxstyleliteralemphasis{\sphinxupquote{{[}}}\sphinxstyleliteralemphasis{\sphinxupquote{float}}\sphinxstyleliteralemphasis{\sphinxupquote{{]}}}) – тейк\sphinxhyphen{}профит

\item {} 
\sphinxAtStartPar
\sphinxstyleliteralstrong{\sphinxupquote{comment}} (\sphinxstyleliteralemphasis{\sphinxupquote{Optional}}\sphinxstyleliteralemphasis{\sphinxupquote{{[}}}\sphinxstyleliteralemphasis{\sphinxupquote{str}}\sphinxstyleliteralemphasis{\sphinxupquote{{]}}}) – комментарий, описание действия, которое нужно совершить

\end{itemize}

\end{description}\end{quote}

\end{fulllineitems}


\end{fulllineitems}

\index{TypeAction (class in src.structures.st\_strategies)@\spxentry{TypeAction}\spxextra{class in src.structures.st\_strategies}}

\begin{fulllineitems}
\phantomsection\label{\detokenize{src.structures:src.structures.st_strategies.TypeAction}}
\pysigstartsignatures
\pysigline{\sphinxbfcode{\sphinxupquote{class\DUrole{w}{  }}}\sphinxbfcode{\sphinxupquote{TypeAction}}}
\pysigstopsignatures
\sphinxAtStartPar
Bases: \sphinxcode{\sphinxupquote{object}}

\sphinxAtStartPar
Класс, в котором определены флаги действий
\index{BUY (TypeAction attribute)@\spxentry{BUY}\spxextra{TypeAction attribute}}

\begin{fulllineitems}
\phantomsection\label{\detokenize{src.structures:src.structures.st_strategies.TypeAction.BUY}}
\pysigstartsignatures
\pysigline{\sphinxbfcode{\sphinxupquote{BUY}}\sphinxbfcode{\sphinxupquote{\DUrole{w}{  }\DUrole{p}{=}\DUrole{w}{  }1}}}
\pysigstopsignatures
\end{fulllineitems}

\index{NOTHING (TypeAction attribute)@\spxentry{NOTHING}\spxextra{TypeAction attribute}}

\begin{fulllineitems}
\phantomsection\label{\detokenize{src.structures:src.structures.st_strategies.TypeAction.NOTHING}}
\pysigstartsignatures
\pysigline{\sphinxbfcode{\sphinxupquote{NOTHING}}\sphinxbfcode{\sphinxupquote{\DUrole{w}{  }\DUrole{p}{=}\DUrole{w}{  }0}}}
\pysigstopsignatures
\end{fulllineitems}

\index{SELL (TypeAction attribute)@\spxentry{SELL}\spxextra{TypeAction attribute}}

\begin{fulllineitems}
\phantomsection\label{\detokenize{src.structures:src.structures.st_strategies.TypeAction.SELL}}
\pysigstartsignatures
\pysigline{\sphinxbfcode{\sphinxupquote{SELL}}\sphinxbfcode{\sphinxupquote{\DUrole{w}{  }\DUrole{p}{=}\DUrole{w}{  }\sphinxhyphen{}1}}}
\pysigstopsignatures
\end{fulllineitems}


\end{fulllineitems}



\paragraph{Module contents}
\label{\detokenize{src.structures:module-contents}}

\subsection{Submodules}
\label{\detokenize{src:submodules}}

\subsection{src.bot module}
\label{\detokenize{src:src-bot-module}}

\subsection{Module contents}
\label{\detokenize{src:module-src}}\label{\detokenize{src:module-contents}}\index{module@\spxentry{module}!src@\spxentry{src}}\index{src@\spxentry{src}!module@\spxentry{module}}

\chapter{Indices and tables}
\label{\detokenize{index:indices-and-tables}}\begin{itemize}
\item {} 
\sphinxAtStartPar
\DUrole{xref,std,std-ref}{genindex}

\item {} 
\sphinxAtStartPar
\DUrole{xref,std,std-ref}{modindex}

\item {} 
\sphinxAtStartPar
\DUrole{xref,std,std-ref}{search}

\end{itemize}


\renewcommand{\indexname}{Python Module Index}
\begin{sphinxtheindex}
\let\bigletter\sphinxstyleindexlettergroup
\bigletter{s}
\item\relax\sphinxstyleindexentry{simulation}\sphinxstyleindexpageref{simulation:\detokenize{module-simulation}}
\item\relax\sphinxstyleindexentry{src}\sphinxstyleindexpageref{src:\detokenize{module-src}}
\item\relax\sphinxstyleindexentry{src.structures.st\_portfolio}\sphinxstyleindexpageref{src.structures:\detokenize{module-src.structures.st_portfolio}}
\item\relax\sphinxstyleindexentry{src.structures.st\_purchase}\sphinxstyleindexpageref{src.structures:\detokenize{module-src.structures.st_purchase}}
\item\relax\sphinxstyleindexentry{src.structures.st\_securities}\sphinxstyleindexpageref{src.structures:\detokenize{module-src.structures.st_securities}}
\item\relax\sphinxstyleindexentry{src.structures.st\_strategies}\sphinxstyleindexpageref{src.structures:\detokenize{module-src.structures.st_strategies}}
\end{sphinxtheindex}

\renewcommand{\indexname}{Index}
\printindex
\end{document}